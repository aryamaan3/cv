% resume.tex
%
% (c) 2002 Matthew Boedicker <mboedick@mboedick.org> (original author) http://mboedick.org
% (c) 2003-2007 David J. Grant <davidgrant-at-gmail.com> http://www.davidgrant.ca
%
% This work is licensed under the Creative Commons Attribution-ShareAlike 3.0 Unported License. To view a copy of this license, visit http://creativecommons.org/licenses/by-sa/3.0/ or send a letter to Creative Commons, 171 Second Street, Suite 300, San Francisco, California, 94105, USA.

\documentclass[letterpaper,10pt]{article}

%-----------------------------------------------------------
%Margin setup

\setlength{\voffset}{0.1in}
\setlength{\paperwidth}{8.5in}
\setlength{\paperheight}{11in}
\setlength{\headheight}{0in}
\setlength{\headsep}{0in}
\setlength{\textheight}{11in}
\setlength{\textheight}{9.5in}
\setlength{\topmargin}{-0.25in}
\setlength{\textwidth}{7in}
\setlength{\topskip}{0in}
\setlength{\oddsidemargin}{-0.25in}
\setlength{\evensidemargin}{-0.25in}
%-----------------------------------------------------------
%\usepackage{fullpage}
\usepackage{shading}
\usepackage{hyperref}
%\textheight=9.0in
\pagestyle{empty}
\raggedbottom
\raggedright
\setlength{\tabcolsep}{0in}

%-----------------------------------------------------------
%Custom commands
\newcommand{\resitem}[1]{\item #1 \vspace{-2pt}}
\newcommand{\resheading}[1]{{\large \parashade[.9]{sharpcorners}{\textbf{#1 \vphantom{p\^{E}}}}}}
\newcommand{\ressubheading}[4]{
\begin{tabular*}{6.5in}{l@{\extracolsep{\fill}}r}
		\textbf{#1} & #2 \\
		\textit{#3} & \textit{#4} \\
\end{tabular*}\vspace{-6pt}}
%-----------------------------------------------------------


\begin{document}

\begin{tabular*}{7in}{l@{\extracolsep{\fill}}r}
\textbf{\Large Aryamaan "Arry" Kunwar} \\
0782404718 &   linkedin.com/in/aryamaan3/\\
aryamaan3@gmail.com & aryamaan3.github.io\\
\end{tabular*}
\\

\vspace{0.1in}

\resheading{Formations}
\begin{itemize}
\item
	\ressubheading{Université Côte d'Azur}{Sophia Antipolis}{Licence MIAGE}{Septembre 2018 - Septembre 2021}
	\begin{itemize}
		\resitem{Informatique, Mathématiques, Economie \& Gestion}
	\end{itemize}

\item
	\ressubheading{Lycée Renoir}{Cagnes Sur Mer}{Baccalauréat Général série scientifique}{2018}

\end{itemize}

%\resheading{Expériences Professionnelles}
%\begin{itemize}
%\item
	%\ressubheading{Coonver Consultancy}{Nice}{Marketing Réseaux Sociaux}{Juin 2019 - Août 2019}
	%\begin{itemize}
		%\resitem{Création et gestion des réseaux sociaux}
	%\end{itemize}
%\end{itemize}

\resheading{Compétences}
\begin{itemize}
\item
	\ressubheading{Bilingue}{}{Ayant comme langue maternelle l'Anglais}{}
\item
	\ressubheading{Programmation}{}{Java, C, Python, R}{}
\item
	\ressubheading{Web}{}{Javascript, PHP, HTML/CSS}{}
\item
	\ressubheading{Frameworks}{}{Angular, Node.js, Maven}{}
\item
	\ressubheading{Data Science}{}{Python(Pandas/Seaborn), R, SQL}{}
\item
	\ressubheading{Informatique}{}{POO, Algorithmes, Bases de données, Systèmes \& Réseaux, Méthodes Agiles(GIT), \LaTeX}{}
\item
	\ressubheading{Mathématiques}{}{Statistiques, Probabilités, Algèbre Linéaire}{}
\item
	\ressubheading{Economie \& Gestion}{}{Finance, Macroéconomie, Microéconomie, Comptabilité, Marketing}{}
	
\end{itemize}

\resheading{Projets Personnelles}
\begin{itemize}
\item
	\ressubheading{Jeu sur Java}{\href{https://github.com/aryamaan3/Batisseurs}{Code source }{et}	\href{https://aryamaan3.github.io/Batisseurs/}{Demo}{ sur GitHub}}{Création d'un jeu avec implementation client-serveur}{Java, JUnit, Maven, Git, GitHub}
	\begin{itemize}
		\resitem{Projet en groupe de 5}
		\resitem{Utilisations des méthodes agiles avec des livraisons régulières pendant 9 semaines}
		\resitem{Utilisation des principes GRASP et SOLID pour l'architecture}
		
	\end{itemize}
\item	
	\ressubheading{Estimation du prix des biens immobiliers}{Kaggle}{Réalisé sur Jupyter notebook}{Python, Pandas, Seaborn, NumPy}
	\begin{itemize}
		\resitem{Fichier csv avec des centaines de maisons et leurs attributs}
		\resitem{Nettoyage et analyse des données}
		\resitem{Visualisation de données}
	\end{itemize}
\item	
	\ressubheading{Interface méteo}{\href{https://github.com/aryamaan3/Weather-Interface}{Code source sur GitHub}}{Génération du plan avec la météo d'une adresse}{Javascript, JSON, PHP}
	\begin{itemize}
		\resitem{Gestion du JSON envoyé par un api de plans et de météo}
		\resitem{Gestion côté serveur}	
	\end{itemize}
\end{itemize}

\end{document}